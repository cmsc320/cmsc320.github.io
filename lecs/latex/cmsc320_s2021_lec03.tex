\documentclass{beamer}

\usepackage{amssymb}
\usepackage{fancyvrb}
\usepackage{stmaryrd}
\usepackage{graphicx}
\usefonttheme{serif}


\newcommand{\Nat}{\mathbb{N}}

\title{Data Science}%\texorpdfstring{$\mathbb{N}$}}
\subtitle{CMSC 320}
\date{Febuary 1\textsuperscript{st}, 2021}

\usetheme{jmct}

\usepackage{calc}

\newcommand{\textover}[3][l]{%
 % #1 is the alignment, default l
 % #2 is the text to be printed
 % #3 is the text for setting the width
 \makebox[\widthof{#3}][#1]{#2}%
 }

\newcommand{\blueit}[1]{%
  {\color{dark-lucid-blue}#1}%
}
\newcommand{\blueite}[1]{%
  \blueit{\emph{#1}}%
}


\newcommand{\myquote}[3]{
  ``#1''
  \vspace{3pt}
  \hrule
  \begin{flushright}
  --- \blueit{\emph{#2}}, \emph{#3}
  \end{flushright}
}

\begin{document}
	\frame {
		\titlepage
	}

%%%%%%%%%%%%%%%%%%%%%%%%%%%%%%%%%%%%%%%% 
%%% Intro
%%%%%%%%%%%%%%%%%%%%%%%%%%%%%%%%%%%%%%%% 

  \frame{
    \frametitle{This Lecture}
    Python?
  }

  \frame{
    \frametitle{Before we start...}
      \begin{enumerate}
        \item<2 - 4> Quizzes.
        \item<3 - 4> Office hours.
        \item<4 - 4> How to ask questions.
      \end{enumerate}
  }

  \frame{
    \frametitle{Quizzes}
      \onslide<2>{You only need to take 10, don't panic if you miss one.}

      \onslide<3>{If you still have concerns, email me (if you emailed me \blueit{before} my email, send me another)}
  }

  \frame{
    \frametitle{Office Hours}
    \onslide<2>{\begin{itemize}
      \item We're going to use Quuly
      \item Quuly doesn't work in every country
      \item If this affects you, let me know
    \end{itemize}
    }
  }

  \frame{
    \frametitle{How to ask questions}
    \onslide<2>{This class is about communication!}
    \onslide<3>{\begin{itemize}
      \item Explaining what you've tried already: Good.
      \item Screenshots: Bad... usually.
    \end{itemize}
    }
  }


%%%%%%%%%%%%%%%%%%%%%%%%%%%%%%%%%%%%%%%% 
%%% Python
%%%%%%%%%%%%%%%%%%%%%%%%%%%%%%%%%%%%%%%% 

  \frame{
    \frametitle{What's in a language?}
    \begin{itemize}
      \item<2-> Languages can be (roughly) organized into \blueit{paradigms}
      \item<3-> Python is multi-paradigm, but leans on the imperative and OO
      \item<4-> These paradigms and languages often have \blueit{idioms}
      \item<5-> A big part of becoming comfortable in a new PL is learning its idioms
    \end{itemize}

  }

  \frame{
    \frametitle{Python.}
    \onslide<2->{Python tends to emphasize the following:}
    \begin{itemize}
      \item<3-> Simple code
      \item<4-> Being explicit
      \item<5-> Working on flat data-structures when possible
      \item<6-> Emphasize readability (code is also communication!)
    \end{itemize}
  }


  \frame{
    \frametitle{iPython (which became part of Jupyter).}
    \onslide<2->{Code alone is great, but what about explanation}
    \begin{itemize}
      \item<3-> Documenting code is good and necessary, but if you want to \blueit{show what the code is doing}, it leaves something to be desired.
      \item<4-> `Notebooks' are meant to address this: Show the code and what it produces, all in the same document
      \item<5-> This is \blueit{essential} for data-science as the code is often the least important thing!
    \end{itemize}
  }

  \frame{
    \frametitle{Learning Python, today}
    \onslide<2->{The vast majority of your programming skills will transfer easily (soapbox: because syntax isn't the main thing!)}
    \onslide<3->{Things we will cover today:}
    \begin{itemize}
      \item<4-> Using the repl
      \item<5-> Defining functions
      \item<6-> Counting and interating
      \item<7-> map and filter
      \item<8-> Iteration cooked two ways
      \item<9-> List comprehensions (ask me how I feel about them)
    \end{itemize}
  }

  \frame{
    \frametitle{Python 2 vs Python 3}
    \onslide<2->{It's been a wild ride.}
    \begin{itemize}
      \item<3-> Most differences are minor
      \item<4-> Some differences break compatibility (code for one won't work for the other)
      \item<5-> For better or for worse (matter of opinion...) Python 3 is the medium-to-long-term future
      \item<6-> Very little reason to \blueit{start} new projects in Python 2.
    \end{itemize}
  }



%%%%%%%%%%%%%%%%%%%%%%%%%%%%%%%%%%%%%%%% 
%%% Conclusion
%%%%%%%%%%%%%%%%%%%%%%%%%%%%%%%%%%%%%%%% 

  \frame{
    \frametitle{Any Questions?}
  }

  \frame{
    \frametitle{Closing thoughts}
    \onslide<2-> {This class is not a Python class.}
    \onslide<3-> {That said, use this time to learn Python!}
    \onslide<4-> {But just know that nothing we learn about \blueit{Data Science} requires Python}
  }

  \frame{
    \frametitle{}
    Thanks for your time!

  }

\end{document}
