\documentclass{beamer}

\usepackage{amssymb}
\usepackage{fancyvrb}
\usepackage{stmaryrd}
\usepackage{graphicx}
\usefonttheme{serif}


\newcommand{\Nat}{\mathbb{N}}

\title{Numpy}%\texorpdfstring{$\mathbb{N}$}}
\subtitle{Numerical Computing in Python}
\date{Febuary 10\textsuperscript{st}, 2021}

\usetheme{jmct}

\usepackage{calc}

\newcommand{\textover}[3][l]{%
 % #1 is the alignment, default l
 % #2 is the text to be printed
 % #3 is the text for setting the width
 \makebox[\widthof{#3}][#1]{#2}%
 }

\newcommand{\blueit}[1]{%
  {\color{dark-lucid-blue}#1}%
}
\newcommand{\blueite}[1]{%
  \blueit{\emph{#1}}%
}


\newcommand{\myquote}[3]{
  ``#1''
  \vspace{3pt}
  \hrule
  \begin{flushright}
  --- \blueit{\emph{#2}}, \emph{#3}
  \end{flushright}
}

\begin{document}
	\frame {
		\titlepage
	}

%%%%%%%%%%%%%%%%%%%%%%%%%%%%%%%%%%%%%%%% 
%%% Intro
%%%%%%%%%%%%%%%%%%%%%%%%%%%%%%%%%%%%%%%% 

  \frame{
    \frametitle{This Lecture}
    You got \blueit{num}bers? We got \blueit{py}thon.
  }

  \frame{
    \frametitle{Before we start...}
      \begin{enumerate}
        \item<2 -> Assignment tomorrow.
        \item<3 -> Say hello to the mod(s).
        \item<4 -> Quick web-scraping demo
      \end{enumerate}
  }

  \frame{
    \frametitle{Assignment}
      \begin{enumerate}
        \item<2 -> Going live tomorrow (Feb 11).
        \item<3 -> You will have \blueit{THREE} weeks.
        \item<4 -> We have not covered all of the material yet, but we will before the due date.
        \item<5 -> Do the parts you can as we progress.
      \end{enumerate}
  }

  \frame{
    \frametitle{Mod(s)}
      We're gonna try having TAs monitor the chat so that I can focus on presenting. This is what I've told them:
      \begin{enumerate}
        \item<2 -> Chatting and banter is okay
        \item<3 -> Any form of disrespect/gatekeeping is not okay (I'd rather have false-positives than false negatives)
        \item<4 -> They'll answer questions in chat, but will interrupt me if it needs my attention.
        \item<5 -> If it's on-topic, urgent, and requires my attention, you should still feel free to interrupt me.
      \end{enumerate}
  }

  \frame{
    \frametitle{Quick web-scraping demo}
      To the Notebook.
  }


%%%%%%%%%%%%%%%%%%%%%%%%%%%%%%%%%%%%%%%% 
%%% Data
%%%%%%%%%%%%%%%%%%%%%%%%%%%%%%%%%%%%%%%% 

  \frame{
    \frametitle{Numpy}
    Numpy is a very popular library in Python. It is one of many data-focused libraries we will use:

      \begin{enumerate}
        \item<2 -> Numpy
        \item<3 -> Pandas
        \item<4 -> Matplotlib
        \item<5 -> various analysis and ML libraries
      \end{enumerate}

  }

  \frame{
    \frametitle{The domain of Numpy}
    Numpy is for working with $n$-dimensional array objects. 
    \onslide<2->{This includes working with these \blueit{in Python} and \blueit{calling out of Python} to C/C++/Fortran/etc. code.}
  }

  \frame{
    \frametitle{Why Numpy}
      In Python, Numpy is the industry standard:
      \begin{enumerate}
        \item<2 -> Provides many of the basic functions: iteration, Fourier, PRNGs, etc.
        \item<3 -> Has well-understood `escape hatches' for when you want to use functionality implemented in a different language.
        \item<4 -> Many of the other libraries we will use this semester work with Numpy objects out of the box.
      \end{enumerate}
  }

  \frame{
    \frametitle{Main thing}
      Numpy provides the ndarray object:
      \begin{enumerate}
        \item<2 -> Fixed size (pros/cons?)
        \item<3 -> Homogeneous (pros/cons?)
        \item<4 -> Heavily optimized
      \end{enumerate}
  }

  \begin{frame}[fragile]
    \frametitle{Other things}
      \begin{enumerate}
        \item<2 -> Many integer types (\verb-intc-, \verb`int{8|16|32|64}`, \verb`float{16|32|64}`, complex numbers, booleans, and more!
      \end{enumerate}
  \end{frame}

  \frame{
    \frametitle{To the Notebook!}
      What the title says.
  }


%%%%%%%%%%%%%%%%%%%%%%%%%%%%%%%%%%%%%%%% 
%%% Conclusion
%%%%%%%%%%%%%%%%%%%%%%%%%%%%%%%%%%%%%%%% 

  \frame{
    \frametitle{Thanks for your time!}
  }

\end{document}
